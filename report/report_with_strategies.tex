\documentclass{article}%
\usepackage[T1]{fontenc}%
\usepackage[utf8]{inputenc}%
\usepackage{lmodern}%
\usepackage{textcomp}%
\usepackage{lastpage}%
\usepackage[tmargin=1cm,lmargin=2cm]{geometry}%
\usepackage[cache=false]{minted}
%
%
%
\begin{document}%
\author{Guilhem MARION, Robert KALNA}
\title{Game theory project: strategies for the game Troll\&Castles}
\date{Mai 2018}
\maketitle
\normalsize%
\section{Strategies}

This project shows a way to compute efficient strategies for the Troll and Castle game using game theory and linear programming. The idea is to formlize each of the round with game theory and find the strategies that maximize the worth case using leanar programming. 

The way of doing actually minimize the loss but don't try to maximize the gain. That's why we choose to implement the elimination of dominated strategies. We claim that this way of doing is capable of maximise the gain in the case of a game against a \textit{good player}. Be a \textit{good player} means be able to play the strategies that is the best for you in terme of mathematical expectation. 

Therefore, this way of deciding is the best in term of expectation, in other terms, the strategies computed are the best in convergence, because of the Law of Large numbers. Due to the definition of convergence, we cannot ensure for a finite number of games that our strategy will be better then an other one. That's why some strategies can win the one we compute on 1000 repetitions. 

To try to win faster in some cases we can try to guess the strategy of the other player in term of conditional probabilities (i.e. the probability of playing a strategy depending of the current state). Then, we can compute of distribution in the strategies that maximise the gain and the gain in the worth case. The idea is that allow to win faster and surely with a greather gain in the case that we have a good guess of the strategy of the other player and that he wont change it during the game. 
For that, we can use Machine Learning (Markov Chains of order $n$, Neural Networks, ...) to learn the distribution while we are playing the prudent strategy, and when the guessed distribution is convenging very well use it to choose what to play knowing it. This part is not implemented and can also be very risky because of all the unknown parameters of the system (the player can change strategy, can also use Machine Learning to find a better strategy then us, convergence time can be very long on certain strategies, ...). 

You can find here simulations we made with other strategies. 

\subsection{Random number of stones}

\begin{minted}{python}
def strategy_random(game, previous_parties):
	number_of_stones_of_enemy = min(game.stockGauche, game.stockDroite + 1)
	return int(np.random.choice(range(1, number_of_stones_of_enemy + 1)))
\end{minted}

\subsection{Always throw 2 stones}

\begin{minted}{python}
def strategy_always_throw_two(game, previous_parties):
	number_of_stones = game.stockGauche
	return min(2, number_of_stones)
\end{minted}

\subsection{Gaussian with location of 2 and variance of 0.5}

\begin{minted}{python}
def strategy_gaussian(game, previous_parties):
	stones_to_throw = np.random.normal(2, 0.5)
	if stones_to_throw > game.stockGauche:
		stones_to_throw = min(np.random.normal(game.stockGauche//2, 3), game.stockGauche)
	return int(max(stones_to_throw, 1))
\end{minted}

\subsection{Strategy leading to Nash equilibrium}

Distributions have been calculated before and stored in pickles: \texttt{distributions} is an object loaded from a pickle. There is one for games with 7 fields and another with 15 fields. The tables of the utilities have also been calculated and stored in pickles, see \texttt{field7/utilities.pkl} and \texttt{field15/utilities.pkl}.

\begin{minted}{python}
def strategy_of_nash(game, previous_parties):
	troll_position = int(game.positionTroll - (game.nombreCases - 1) // 2)
	stones_left = game.stockGauche
	stones_right = game.stockDroite
	if (stones_left, stones_right, troll_position) in distributions:
		((distribution, distribution_ind), g) = distributions[
					stones_left, stones_right, troll_position]
	else:
		((distribution, distribution_ind), g) = db.calculate_what_to_play(
e				stones_left, stones_right, troll_position)
	distribution = np.array(distribution)
	distribution /= distribution.sum()
	X = np.random.choice(distribution_ind, 1, p=distribution)
	return int(X[0])
\end{minted}

\subsection{Eager version of the strategy leading to Nash equilibrium}

\begin{minted}{python}
def strategy_nash_eager(game, previous_parties):
	troll_position = int(game.positionTroll - (game.nombreCases - 1) // 2)
	stones_left = game.stockGauche
	stones_right = game.stockDroite
	if (stones_left, stones_right, troll_position) in distributions:
		((distribution, distribution_ind), g) = distributions[
					stones_left, stones_right, troll_position]
	else:
		((distribution, distribution_ind), g) = db.calculate_what_to_play(
					stones_left, stones_right, troll_position)
	return int(np.array(distribution).argmax() + 1)
\end{minted}

\section{Number of fields: 7, stones: 15}%
\label{sec:Number of fields 7, stones 15}%
\subsection{Strategy of nash VS random number of stones}%
\label{subsec:Strategy of nash VS random number of stones}%
{-}{-}{-}{-}{-} Resultats de la simulation {-}{-}{-}{-}{-}\newline%
		\newline%
Matchs prevus : 1000\newline%
Matchs joues : 1000\newline%
\newline%
Victoires du joueur de gauche : 951\newline%
Victoires du joueur de droite : 42\newline%
Matchs nuls : 7\newline%
\newline%
Victoire du joueur de gauche !

%
\subsection{Strategy of nash VS eager version of strategy of nash}%
\label{subsec:Strategy of nash VS eager version of strategy of nash}%
{-}{-}{-}{-}{-} Resultats de la simulation {-}{-}{-}{-}{-}\newline%
		\newline%
Matchs prevus : 1000\newline%
Matchs joues : 1000\newline%
\newline%
Victoires du joueur de gauche : 275\newline%
Victoires du joueur de droite : 283\newline%
Matchs nuls : 442\newline%
\newline%
Victoire du joueur de droite !

%
\subsection{Strategy of nash VS gaussian with location of 2 and variance of 0.5}%
\label{subsec:Strategy of nash VS gaussian with location of 2 and variance of 0.5}%
{-}{-}{-}{-}{-} Resultats de la simulation {-}{-}{-}{-}{-}\newline%
		\newline%
Matchs prevus : 1000\newline%
Matchs joues : 1000\newline%
\newline%
Victoires du joueur de gauche : 675\newline%
Victoires du joueur de droite : 275\newline%
Matchs nuls : 50\newline%
\newline%
Victoire du joueur de gauche !

%
\subsection{Strategy of nash VS always throw two stones}%
\label{subsec:Strategy of nash VS always throw two stones}%
{-}{-}{-}{-}{-} Resultats de la simulation {-}{-}{-}{-}{-}\newline%
		\newline%
Matchs prevus : 1000\newline%
Matchs joues : 1000\newline%
\newline%
Victoires du joueur de gauche : 342\newline%
Victoires du joueur de droite : 378\newline%
Matchs nuls : 280\newline%
\newline%
Victoire du joueur de droite !

%
\subsection{Nash equilibrium}%
\label{subsec:Nash equilibrium}%
{-}{-}{-}{-}{-} Resultats de la simulation {-}{-}{-}{-}{-}\newline%
		\newline%
Matchs prevus : 1000\newline%
Matchs joues : 1000\newline%
\newline%
Victoires du joueur de gauche : 309\newline%
Victoires du joueur de droite : 324\newline%
Matchs nuls : 367\newline%
\newline%
Victoire du joueur de droite !

%
\section{Number of fields: 7, stones: 30}%
\label{sec:Number of fields 7, stones 30}%
\subsection{Strategy of nash VS random number of stones}%
\label{subsec:Strategy of nash VS random number of stones}%
{-}{-}{-}{-}{-} Resultats de la simulation {-}{-}{-}{-}{-}\newline%
		\newline%
Matchs prevus : 1000\newline%
Matchs joues : 1000\newline%
\newline%
Victoires du joueur de gauche : 947\newline%
Victoires du joueur de droite : 50\newline%
Matchs nuls : 3\newline%
\newline%
Victoire du joueur de gauche !

%
\subsection{Strategy of nash VS eager version of strategy of nash}%
\label{subsec:Strategy of nash VS eager version of strategy of nash}%
{-}{-}{-}{-}{-} Resultats de la simulation {-}{-}{-}{-}{-}\newline%
		\newline%
Matchs prevus : 1000\newline%
Matchs joues : 1000\newline%
\newline%
Victoires du joueur de gauche : 438\newline%
Victoires du joueur de droite : 457\newline%
Matchs nuls : 105\newline%
\newline%
Victoire du joueur de droite !

%
\subsection{Strategy of nash VS gaussian with location of 2 and variance of 0.5}%
\label{subsec:Strategy of nash VS gaussian with location of 2 and variance of 0.5}%
{-}{-}{-}{-}{-} Resultats de la simulation {-}{-}{-}{-}{-}\newline%
		\newline%
Matchs prevus : 1000\newline%
Matchs joues : 1000\newline%
\newline%
Victoires du joueur de gauche : 732\newline%
Victoires du joueur de droite : 268\newline%
Matchs nuls : 0\newline%
\newline%
Victoire du joueur de gauche !

%
\subsection{Strategy of nash VS always throw two stones}%
\label{subsec:Strategy of nash VS always throw two stones}%
{-}{-}{-}{-}{-} Resultats de la simulation {-}{-}{-}{-}{-}\newline%
		\newline%
Matchs prevus : 1000\newline%
Matchs joues : 1000\newline%
\newline%
Victoires du joueur de gauche : 617\newline%
Victoires du joueur de droite : 383\newline%
Matchs nuls : 0\newline%
\newline%
Victoire du joueur de gauche !

%
\subsection{Nash equilibrium}%
\label{subsec:Nash equilibrium}%
{-}{-}{-}{-}{-} Resultats de la simulation {-}{-}{-}{-}{-}\newline%
		\newline%
Matchs prevus : 1000\newline%
Matchs joues : 1000\newline%
\newline%
Victoires du joueur de gauche : 451\newline%
Victoires du joueur de droite : 491\newline%
Matchs nuls : 58\newline%
\newline%
Victoire du joueur de droite !

%
\section{Number of fields: 15, stones: 30}%
\label{sec:Number of fields 15, stones 30}%
\subsection{Strategy of nash VS random number of stones}%
\label{subsec:Strategy of nash VS random number of stones}%
{-}{-}{-}{-}{-} Resultats de la simulation {-}{-}{-}{-}{-}\newline%
		\newline%
Matchs prevus : 1000\newline%
Matchs joues : 1000\newline%
\newline%
Victoires du joueur de gauche : 1000\newline%
Victoires du joueur de droite : 0\newline%
Matchs nuls : 0\newline%
\newline%
Victoire du joueur de gauche !

%
\subsection{Strategy of nash VS eager version of strategy of nash}%
\label{subsec:Strategy of nash VS eager version of strategy of nash}%
{-}{-}{-}{-}{-} Resultats de la simulation {-}{-}{-}{-}{-}\newline%
		\newline%
Matchs prevus : 1000\newline%
Matchs joues : 1000\newline%
\newline%
Victoires du joueur de gauche : 0\newline%
Victoires du joueur de droite : 1\newline%
Matchs nuls : 999\newline%
\newline%
Victoire du joueur de droite !

%
\subsection{Strategy of nash VS gaussian with location of 2 and variance of 0.5}%
\label{subsec:Strategy of nash VS gaussian with location of 2 and variance of 0.5}%
{-}{-}{-}{-}{-} Resultats de la simulation {-}{-}{-}{-}{-}\newline%
		\newline%
Matchs prevus : 1000\newline%
Matchs joues : 1000\newline%
\newline%
Victoires du joueur de gauche : 476\newline%
Victoires du joueur de droite : 214\newline%
Matchs nuls : 310\newline%
\newline%
Victoire du joueur de gauche !

%
\subsection{Strategy of nash VS always throw two stones}%
\label{subsec:Strategy of nash VS always throw two stones}%
{-}{-}{-}{-}{-} Resultats de la simulation {-}{-}{-}{-}{-}\newline%
		\newline%
Matchs prevus : 1000\newline%
Matchs joues : 1000\newline%
\newline%
Victoires du joueur de gauche : 1\newline%
Victoires du joueur de droite : 0\newline%
Matchs nuls : 999\newline%
\newline%
Victoire du joueur de gauche !

%
\subsection{Nash equilibrium}%
\label{subsec:Nash equilibrium}%
{-}{-}{-}{-}{-} Resultats de la simulation {-}{-}{-}{-}{-}\newline%
		\newline%
Matchs prevus : 1000\newline%
Matchs joues : 1000\newline%
\newline%
Victoires du joueur de gauche : 4\newline%
Victoires du joueur de droite : 2\newline%
Matchs nuls : 994\newline%
\newline%
Victoire du joueur de gauche !

%
\section{Number of fields: 15, stones: 50}%
\label{sec:Number of fields 15, stones 50}%
\subsection{Strategy of nash VS random number of stones}%
\label{subsec:Strategy of nash VS random number of stones}%
{-}{-}{-}{-}{-} Resultats de la simulation {-}{-}{-}{-}{-}\newline%
		\newline%
Matchs prevus : 1000\newline%
Matchs joues : 1000\newline%
\newline%
Victoires du joueur de gauche : 1000\newline%
Victoires du joueur de droite : 0\newline%
Matchs nuls : 0\newline%
\newline%
Victoire du joueur de gauche !

%
\subsection{Strategy of nash VS eager version of strategy of nash}%
\label{subsec:Strategy of nash VS eager version of strategy of nash}%
{-}{-}{-}{-}{-} Resultats de la simulation {-}{-}{-}{-}{-}\newline%
		\newline%
Matchs prevus : 1000\newline%
Matchs joues : 1000\newline%
\newline%
Victoires du joueur de gauche : 303\newline%
Victoires du joueur de droite : 243\newline%
Matchs nuls : 454\newline%
\newline%
Victoire du joueur de gauche !

%
\subsection{Strategy of nash VS gaussian with location of 2 and variance of 0.5}%
\label{subsec:Strategy of nash VS gaussian with location of 2 and variance of 0.5}%
{-}{-}{-}{-}{-} Resultats de la simulation {-}{-}{-}{-}{-}\newline%
		\newline%
Matchs prevus : 1000\newline%
Matchs joues : 1000\newline%
\newline%
Victoires du joueur de gauche : 734\newline%
Victoires du joueur de droite : 254\newline%
Matchs nuls : 12\newline%
\newline%
Victoire du joueur de gauche !

%
\subsection{Strategy of nash VS always throw two stones}%
\label{subsec:Strategy of nash VS always throw two stones}%
{-}{-}{-}{-}{-} Resultats de la simulation {-}{-}{-}{-}{-}\newline%
		\newline%
Matchs prevus : 1000\newline%
Matchs joues : 1000\newline%
\newline%
Victoires du joueur de gauche : 177\newline%
Victoires du joueur de droite : 65\newline%
Matchs nuls : 758\newline%
\newline%
Victoire du joueur de gauche !

%
\subsection{Nash equilibrium}%
\label{subsec:Nash equilibrium}%
{-}{-}{-}{-}{-} Resultats de la simulation {-}{-}{-}{-}{-}\newline%
		\newline%
Matchs prevus : 1000\newline%
Matchs joues : 1000\newline%
\newline%
Victoires du joueur de gauche : 330\newline%
Victoires du joueur de droite : 355\newline%
Matchs nuls : 315\newline%
\newline%
Victoire du joueur de droite !

%
\end{document}
